%評価・考察
 本章ではV字モデルに従って実装したシステムを評価することにより、本システムに対する利点と問題点について述べる。

 このシステムの利点としてYoloを使用することによる複数のバーコード領域特定がある。

 今回実装と検証を行って一番の問題点となったのがWebカメラが撮影した商品の距離によって検知率が大きく異なる点である。今回使用したWebカメラはロジクール製のC615を使用している。こちらのモデルを選ぶ前に同メーカーのC270nを使用していたが5cm程の近距離でバーコードを撮影してもバーコード番号の識別が不可能であった。ここでいう識別はpyzbarによる番号の識別でありYoloによる領域判定は検知していた。C270nモデルの画質では判別が不可能と判明したためC615モデルに変更したところ約10cmの範囲の近距離ならばpyzbarで正確にすべて識別することがテストを行い判明した。しかし、C615モデルで10cm以上離すと検知しなくなった。画像に対して2値化処理とノイズ除去を行ったが、元の画像のバーコードがぼやけてバーの境界があやふやになっている状態だったため結果は変わらないという結果に至った。代わりにiPhone5cで15cmの距離で撮影したバーコードを映した動画をテストしたところpyzbarでの識別をある程度行うことができたため、本システムはWebカメラの画質によって大きく識別精度が変わる問題点がある。