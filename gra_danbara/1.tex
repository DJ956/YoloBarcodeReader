%まえがき
本論文では、番号からラズベリーパイと各種センサとサーバを組み合わせて商品識別システムを構築した。皆さんもご存じの通り現在の日本は少子高齢化の流れで人的資源が減少している。総務省の調べでは総人口が2008年をピークに減少している\cite{population}.人的資源の減少という問題は私たちの身近なスーパーマーケットや小売店にも表れており、その対処法として決済の一部をユーザに一部負担してもらうセルフレジなどの導入が進んでいる。またAmazonなどの大企業では無人店舗AmazonGo\cite{amazongo}.をアメリカの一部地域で展開しているがそのサービスを稼働させるには高い技術力と資金力が必要となる。先ほど述べたセルフレジの金額も登録機と精算機を合わせて平均で300万を超えるものが多い\cite{self_register}.セルフレジの導入は人手不足問題を抱える店舗においてメリットも大きいが負担も大きい問題がある。そこでこのコスト面での問題を解決するために資金力を持たない店舗でも導入しやすい安価なシステムの構築を本研究の目的とした。

本研究では商品の識別から決算までのシステム構築をV字モデルに従い、グループ(段原丞治、真鍋樹)で研究を行った。
要求定義や設計はUML図を用いて定義しそれをもとに開発を行う。実装に関してはエッジ処理側とサーバ処理側で担当を分けた。

本論文は以下のような構成をとる。第2章では用語、実験環境、システム概要の説明を述べる。第3章ではUML図を用いてシステムの要求定義、設計を述べる。第4章では実装内容と検証結果を示す。第5章では実装したシステムの評価及び考察を述べる。第6章では本研究のまとめと今後の課題を示す。