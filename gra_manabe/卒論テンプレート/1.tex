%まえがき
%1-1要約:第一文-論文の趣旨,後1~2文①考えたこと②やったこと③結果の概要④成果の意義
%1-2問題設定(序論):研究をなぜ実施しなければならないのかを書く
%①あなたが解こうとする問題は何ですか?②その問題には必要性や需要はありますか?③あなたの取り組む問題は未解明で新規のものですか?④有用性はどれくらい見込めますか?
%1-3関連研究を引用.1)定石とすべき先行研究 2)定石化されていない部分の分岐状況
%各質問に対する回答パターンについては本参照



本論文では,Webカメラと超音波センサ,ロードセルなどのセンサを用い,安価な商品識別システムの開発を行った.

近年の日本において,少子高齢化の進行により,生産年齢人口は1995年をピークに減少に転じており,総人口も2008年をピークに減少に転じている\cite{population}.生産年齢人口の減少という問題は,スーパーマーケットにも顕著に表れており,セミセルフレジの導入や,無人レジ店舗の導入が進んでいる.しかしながら,


本研究の目的は,既存の無人レジ店舗のような複雑で高価なシステムではなく,中小店でも導入できる安価なシステムの作成である.

本研究ではグループ(段原 丞治,真鍋 樹)で,V字開発モデルに従って商品識別システムの開発を行った.要求分析,基本設計,詳細設計の際はUMLを用いた.