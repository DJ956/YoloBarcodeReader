%第5-1章:評価
%表を作成して,比較評価.評価軸を設定し.表にして比較.定量的評価は数値,定性的評価は〇△×で.


\section{評価}

本システムを評価する.3章で設定した,3点の基本の評価軸より評価する.評価したものを書き表\ref{hyouka}に示す.

\begin{table}[htb]
\begin{center}
\caption{システムの評価}
\begin{tabular}{|c|c|} \hline
基本の評価軸 & 評価 \\ \hline \hline
従来のセルフレジよりコストは抑えられるか & 〇 \\
既存の中小店でも導入が容易か & △ \\
従来のセルフレジより簡単な動作で決済まで行えるか & △\\ \hline
\end{tabular}
\label{hyouka}
\end{center}
\end{table}

表\ref{hyouka}を上から順に説明する.従来のセルフレジよりコストは抑えられるかという評価軸について,2.2節で述べた表\ref{taisho}のスーパーマーケットを対象にして確認をする.Raspberry Piの価格は5,700円程度,各種センサと周辺機器の合計価格は3,500円程度のため,カゴにかかる合計価格は9,200円とする.サーバと周辺機器にかかる価格を約150,000円とする.サーバ1台約150,000円とカゴ90個約828,000円とすると,本システムでかかる価格は約978,000円となり,従来のセルフレジとして2.2節で仮定した登録機1台と精算機7台の合計価格の約5\%程の価格となることが分かった.上記の理由から,従来のセルフレジよりコストを抑えられるとした.

次に,既存の中小店でも導入が容易かという評価軸においては,現段階では容易ではない.Raspberry Piや各種センサがしっかりと固定されておらず,誰でも導入ができるわけではない.しかし,これからしっかり固定できるような状況ができれば,既存の買い物カゴに設置できる規模感であるため可能性があるという観点から△とした.

次に,従来のセルフレジより簡単な動作で決済まで行えるかという評価軸においては,現時点では,商品のバーコードを読み取らせるために商品を回転させ,バーコードリーダを操作する必要はないが,バーコードをWebカメラに向けて台に置く必要があるため,△とした.